\documentclass[a4paper,10pt]{article}

\usepackage{amsmath,amssymb, amsthm}
\usepackage{charter}
%\usepackage{calc}

\usepackage{fullpage}

\usepackage{multirow}
%\usepackage[cp1250]{inputenc}
%\usepackage[T1]{fontenc}
%\usepackage{calligra}
%\usepackage[slovak]{babel}
\usepackage{amsfonts}
\usepackage{layout}
\usepackage[dvips]{graphicx}
\usepackage{url}
\usepackage{color}
\usepackage{graphicx}
\usepackage{algorithmic}
\usepackage{algorithm}
\usepackage{verbatim}
\usepackage{epsfig}
\usepackage[colorlinks,
            linkcolor=red,
            anchorcolor=blue,
            citecolor=blue
            ]{hyperref}
\usepackage{xcolor}

\renewcommand{\algorithmicrequire}{\textbf{Input:}}
\renewcommand{\algorithmicensure}{\textbf{Output:}}

\newcommand{\hlight}[2]{\noindent\colorbox{#1}{%
    \parbox{\dimexpr\linewidth-1\fboxsep}% a box with line-breaks that's just wide enough
        {#2%
        }}}

\newcommand{\HRule}{\rule{\linewidth}{0.5mm}}
 \newcommand{\xtilde}{\tilde x}
 \newcommand{\vtilde}{\tilde v}
 \newcommand{\Embb}{\E}
 \newcommand{\xbar}{\bar x}
 \newcommand{\ie}{\textit{i.e.}}
 
\newcommand{\main}[1]{\footnotesize\textbf{#1}} 
\newcommand{\todo}[1]{{\color{red}#1}}
\newcommand\tagthis{\addtocounter{equation}{1}\tag{\theequation}}


\newcommand{\del}[1]{{\color{red}#1}}

\let\la=\langle
\let\ra=\rangle

\newcommand{\st}{\;:\;}
\newcommand{\ve}[2]{\langle #1 ,  #2 \rangle}

\newcommand{\eqdef}{\stackrel{\text{def}}{=}}

\newcommand{\ii}{{}^{(i)}}

\newcommand{\R}{\mathbb{R}}
\newcommand{\Prob}{\mathbf{Prob}}
\newcommand{\E}{\mathbb{E}}
\newcommand{\Q}{\mathbb{Q}}
\newcommand{\Z}{\mathbb{Z}}

\newcommand{\vc}[2]{#1^{(#2)}}
\newcommand{\nc}[2]{{\color{red}\|#1\|_{(#2)}}}
\newcommand{\ncs}[2]{\|#1\|^2_{(#2)}}
\newcommand{\ncc}[2]{{\color{red}\|#1\|^*_{(#2)}}}
\newcommand{\ls}[1]{{\color{red} \mathcal S(#1)}}
\newcommand{\Rw}[2]{\mathcal R_{#1}(#2)}
\newcommand{\Rws}[2]{\mathcal R^2_{#1}(#2)}
\newcommand{\m}[1]{~\mbox{#1}~}

\newcommand{\nbp}[2]{\|#1\|_{(#2)}}   % norm block primal
\newcommand{\nbd}[2]{\|#1\|_{(#2)}^*} % norm block dual

\newcommand{\lf}{\mathcal L}
\newcommand{\U}{U}
\newcommand{\N}{N}
\newcommand{\mc}[1]{\mathcal #1}

\newcommand{\mLi}{{\color{red}m^{(i)}}}
\newcommand{\gLi}{{\color{red}g^{(i)}}}
%\newcommand{\TLi}{{\color{red}T_L^{(i)}}}
\newcommand{\TLi}[1]{{\color{blue}T^{(#1)}}}

\newcommand{\Lip}{L}

\newcommand{\Rc}[1]{{\color{red}  \mathbf{RC}_{(#1)}}}
\newcommand{\NRCDM}{{\color{red}NRCDM}\  }
\newcommand{\nnz}[1]{{\color{red}\|#1\|_0}}
% sets
\DeclareMathOperator{\card}{card}       % cardinality of a set
\DeclareMathOperator{\diam}{diam}       % diameter of a set
\DeclareMathOperator{\MVEE}{MVEE}       % minim volume enclosing ellipsoid of a set
\DeclareMathOperator{\vol}{vol}         % volume of a set

\DeclareMathOperator{\prox}{prox}         

% statistical
\DeclareMathOperator{\Exp}{\mathbf{E}}           % expectation
\DeclareMathOperator{\Cov}{Cov}         % covariance
\DeclareMathOperator{\Var}{Var}         % variance
\DeclareMathOperator{\Corr}{Corr}       % correlation

% functions and operators
\DeclareMathOperator{\signum}{sign}     % signum/sign of a scalar
\DeclareMathOperator{\dom}{dom}         % domain
\DeclareMathOperator{\epi}{epi}         % epigraph
\DeclareMathOperator{\Ker}{null}        % nullspace/kernel
\DeclareMathOperator{\nullspace}{null}  % nullpsace
\DeclareMathOperator{\range}{range}     % range
\DeclareMathOperator{\Image}{Im}        % image
\DeclareMathOperator{\argmin}{argmin}        % argmin

% topology
\DeclareMathOperator{\interior}{int}    % interior
\DeclareMathOperator{\ri}{rint}         % relative interior
\DeclareMathOperator{\rint}{rint}       % relative interior
\DeclareMathOperator{\bdry}{bdry}       % boundary
\DeclareMathOperator{\cl}{cl}           % closure

% vectors, matrices
\DeclareMathOperator{\linspan}{span}
\DeclareMathOperator{\linspace}{linspace}
\DeclareMathOperator{\cone}{cone}

\DeclareMathOperator{\tr}{tr}           % trace
\DeclareMathOperator{\rank}{rank}       % rank
\DeclareMathOperator{\conv}{conv}       % convex hull
\DeclareMathOperator{\Diag}{Diag}       % Diag(v) = diagonal matrix with v_i on the diagonal
\DeclareMathOperator{\diag}{diag}       % diag(D) = the diagonal vector of matrix D

\DeclareMathOperator{\Arg}{Arg}         % Argument

%\renewcommand{\qedsymbol}{\ding{114}}


\newtheorem{assumption}{Assumption}
\newtheorem{lemma}{Lemma}
\newtheorem{algorithms}{Algorithm}
\newtheorem{theorem}{Theorem}
\newtheorem{proposition}{Proposition}
\newtheorem{example}{Example}
\newtheorem{remark}{Remark}

\theoremstyle{plain}

\newtheorem{prop}[theorem]{Proposition}
\newtheorem{cor}[theorem]{Corollary}
\newtheorem{lem}[theorem]{Lemma}
\newtheorem{claim}[theorem]{Claim}
%\newtheorem{remark}[theorem]{Remark}

\theoremstyle{definition}

\newtheorem{exercise}[theorem]{Exercise}

\newtheorem{rem}[theorem]{Remark}
\newtheorem{que}[theorem]{Question}
\newtheorem{definition}[theorem]{Definition}

 
\title{A Proposal to Research Cutting Plane Tree Algorithm and Compare it with other Algorithms}
\author{Xi He, Chenxin Ma}
 
\begin{document}
\maketitle
 
\section{Introduction} 

In this project, we will focus on cutting plane for large Mixed Integer Programs (MIPs) in the context of Branch and Cut algorithms. More specifically, we will study an efficient algorithm (KBB) for the Mixed Integer Knapsack Program (MIKP). 
Consider a positive integer $n$. For each $k\in\{1,...,n\}$, let $a_k, c_k\in \mathbb Q$, $l_k\in \mathbb Q\cup \{-\infty\}$, and $u_k\in \mathbb Q\cup \{+\infty\}$. Let $b\in\mathbb Q$, and consider $I\subset \{1,...,n\}$. The Mixed Integer Knapsack Problem (MIKP) can be described as:
\begin{align}
\max &\sum_{i=1}^{n}c_ix_i \notag\\
s.t. \; &\sum_{j=1}^{n} a_{k}x_k\leq b \notag\\
&l_k\leq x_k\leq u_k,\quad\forall k\in\{1,2,..,n\} \notag\\
&x_k\in \mathbb Z, \forall k\in I
\end{align} 
Most modern algorithms for solving Knapsack Problem are based either on branch-and-bound or dynamic programming. However, the most efficient codes seldom make explicit use of Linear Programming. The algorithm we implement is an LP-based branch-and-bound algorithm,  which will either (a) proves MIKP is infeasible, (b) proves MIKP is unbounded or (c) finds an optimal solution to MIKP. According to the literature, the proposed algorithm performs quite well in practice, outperforming the general-use mixed integer programming solver CPLEX. 

The algorithm exploits dominance conditions, which differs from traditional linear-programming based algorithms by allowing feasible solutions to be pruned during the branching phase. The idea is that feasible solutions will only be pruned if either (a) they are not optimal (cost-domination-criteria), or (b) if they are optimal, but somewhere else in the tree it is known that there is another optimal solution. In the computational study, we will try to see how  domination plays an role in this algorithm.


\section{Objectives}
we will implement the algorithm in \cite{fukasawa2011exact}, which contains:
\begin{itemize}
\item a simple eight-step pre-processing procedure in order to reduce an instance of MIKP to another, equivalent
instance of MIKP which is easier to solve.
\item a depth-first-search branch and bound algorithm which always branches on the unique fractional variable. We use a simple linear programming algorithm, a variation of Dantzig’s algorithm \cite{dantzig1957discrete}, which runs in linear time by taking advantage of the fact that variables are sorted by decreasing efficiency.
\item using dominance to improve the branch and bound search.(+++)
\end{itemize}
After implementing it, we will conduct extensive computational experiments. The goal is to reproduce the improvement that the new algorithm brings as in \cite{fukasawa2011exact}. Moreover, we will try to combine different branching strategies and bound improvement to compare the different versions of the algorithm. If possible, we will take a glimpse to use this knapsack cut generation procedure in practice to help solve some classes of mixed integer programming problems.


\section{Research plan}

\bibliographystyle{plain} 
\bibliography{literature}

 
\end{document}
