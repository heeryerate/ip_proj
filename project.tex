\documentclass[a4paper,10pt]{article}

\usepackage{amsmath,amssymb, amsthm}
\usepackage{charter}
%\usepackage{calc}

\usepackage{fullpage}

\usepackage{multirow}
%\usepackage[cp1250]{inputenc}
%\usepackage[T1]{fontenc}
%\usepackage{calligra}
%\usepackage[slovak]{babel}
\usepackage{amsfonts}
\usepackage{layout}
\usepackage[dvips]{graphicx}
\usepackage{url}
\usepackage{color}
\usepackage{graphicx}
\usepackage{algorithmic}
\usepackage{algorithm}
\usepackage{verbatim}
\usepackage{epsfig}
\usepackage[colorlinks,
            linkcolor=red,
            anchorcolor=blue,
            citecolor=blue
            ]{hyperref}
\usepackage{xcolor}

\renewcommand{\algorithmicrequire}{\textbf{Input:}}
\renewcommand{\algorithmicensure}{\textbf{Output:}}

\newcommand{\hlight}[2]{\noindent\colorbox{#1}{%
    \parbox{\dimexpr\linewidth-1\fboxsep}% a box with line-breaks that's just wide enough
        {#2%
        }}}

\newcommand{\HRule}{\rule{\linewidth}{0.5mm}}
 \newcommand{\xtilde}{\tilde x}
 \newcommand{\vtilde}{\tilde v}
 \newcommand{\Embb}{\E}
 \newcommand{\xbar}{\bar x}
 \newcommand{\ie}{\textit{i.e.}}
 
\newcommand{\main}[1]{\footnotesize\textbf{#1}} 
\newcommand{\todo}[1]{{\color{red}#1}}
\newcommand\tagthis{\addtocounter{equation}{1}\tag{\theequation}}


\newcommand{\del}[1]{{\color{red}#1}}

\let\la=\langle
\let\ra=\rangle

\newcommand{\st}{\;:\;}
\newcommand{\ve}[2]{\langle #1 ,  #2 \rangle}

\newcommand{\eqdef}{\stackrel{\text{def}}{=}}

\newcommand{\ii}{{}^{(i)}}

\newcommand{\R}{\mathbb{R}}
\newcommand{\Prob}{\mathbf{Prob}}
\newcommand{\E}{\mathbb{E}}
\newcommand{\Q}{\mathbb{Q}}
\newcommand{\Z}{\mathbb{Z}}

\newcommand{\vc}[2]{#1^{(#2)}}
\newcommand{\nc}[2]{{\color{red}\|#1\|_{(#2)}}}
\newcommand{\ncs}[2]{\|#1\|^2_{(#2)}}
\newcommand{\ncc}[2]{{\color{red}\|#1\|^*_{(#2)}}}
\newcommand{\ls}[1]{{\color{red} \mathcal S(#1)}}
\newcommand{\Rw}[2]{\mathcal R_{#1}(#2)}
\newcommand{\Rws}[2]{\mathcal R^2_{#1}(#2)}
\newcommand{\m}[1]{~\mbox{#1}~}

\newcommand{\nbp}[2]{\|#1\|_{(#2)}}   % norm block primal
\newcommand{\nbd}[2]{\|#1\|_{(#2)}^*} % norm block dual

\newcommand{\lf}{\mathcal L}
\newcommand{\U}{U}
\newcommand{\N}{N}
\newcommand{\mc}[1]{\mathcal #1}

\newcommand{\mLi}{{\color{red}m^{(i)}}}
\newcommand{\gLi}{{\color{red}g^{(i)}}}
%\newcommand{\TLi}{{\color{red}T_L^{(i)}}}
\newcommand{\TLi}[1]{{\color{blue}T^{(#1)}}}

\newcommand{\Lip}{L}

\newcommand{\Rc}[1]{{\color{red}  \mathbf{RC}_{(#1)}}}
\newcommand{\NRCDM}{{\color{red}NRCDM}\  }
\newcommand{\nnz}[1]{{\color{red}\|#1\|_0}}
% sets
\DeclareMathOperator{\card}{card}       % cardinality of a set
\DeclareMathOperator{\diam}{diam}       % diameter of a set
\DeclareMathOperator{\MVEE}{MVEE}       % minim volume enclosing ellipsoid of a set
\DeclareMathOperator{\vol}{vol}         % volume of a set

\DeclareMathOperator{\prox}{prox}         

% statistical
\DeclareMathOperator{\Exp}{\mathbf{E}}           % expectation
\DeclareMathOperator{\Cov}{Cov}         % covariance
\DeclareMathOperator{\Var}{Var}         % variance
\DeclareMathOperator{\Corr}{Corr}       % correlation

% functions and operators
\DeclareMathOperator{\signum}{sign}     % signum/sign of a scalar
\DeclareMathOperator{\dom}{dom}         % domain
\DeclareMathOperator{\epi}{epi}         % epigraph
\DeclareMathOperator{\Ker}{null}        % nullspace/kernel
\DeclareMathOperator{\nullspace}{null}  % nullpsace
\DeclareMathOperator{\range}{range}     % range
\DeclareMathOperator{\Image}{Im}        % image
\DeclareMathOperator{\argmin}{argmin}        % argmin

% topology
\DeclareMathOperator{\interior}{int}    % interior
\DeclareMathOperator{\ri}{rint}         % relative interior
\DeclareMathOperator{\rint}{rint}       % relative interior
\DeclareMathOperator{\bdry}{bdry}       % boundary
\DeclareMathOperator{\cl}{cl}           % closure

% vectors, matrices
\DeclareMathOperator{\linspan}{span}
\DeclareMathOperator{\linspace}{linspace}
\DeclareMathOperator{\cone}{cone}

\DeclareMathOperator{\tr}{tr}           % trace
\DeclareMathOperator{\rank}{rank}       % rank
\DeclareMathOperator{\conv}{conv}       % convex hull
\DeclareMathOperator{\Diag}{Diag}       % Diag(v) = diagonal matrix with v_i on the diagonal
\DeclareMathOperator{\diag}{diag}       % diag(D) = the diagonal vector of matrix D

\DeclareMathOperator{\Arg}{Arg}         % Argument

%\renewcommand{\qedsymbol}{\ding{114}}


\newtheorem{assumption}{Assumption}
\newtheorem{lemma}{Lemma}
\newtheorem{algorithms}{Algorithm}
\newtheorem{theorem}{Theorem}
\newtheorem{proposition}{Proposition}
\newtheorem{example}{Example}
\newtheorem{remark}{Remark}

\theoremstyle{plain}

\newtheorem{prop}[theorem]{Proposition}
\newtheorem{cor}[theorem]{Corollary}
\newtheorem{lem}[theorem]{Lemma}
\newtheorem{claim}[theorem]{Claim}
%\newtheorem{remark}[theorem]{Remark}

\theoremstyle{definition}

\newtheorem{exercise}[theorem]{Exercise}

\newtheorem{rem}[theorem]{Remark}
\newtheorem{que}[theorem]{Question}
\newtheorem{definition}[theorem]{Definition}

 
\title{A Proposal to Research Cutting Plane Tree Algorithm and Compare it with other Algorithms}
\author{Chenxin Ma, Xi He}
 
\begin{document}
\maketitle
 
\section{Introduction} 
A mixed-integer linear program (MILP) is a mathematical program with linear constraints in which a specified subset of the variables are required to 
take on integer values. let a standard form polyhedron be
\begin{equation}
 \mc{P} = \{x\in \R^n| Ax=b, x\geq 0\},
\end{equation}

where $A\in \Q^{m\times n},  b\in \Q^m$. Without loss of generality, we assume that the variables indexed $1$ through $p\leq n$ are the integer 
variables. If we denote $\mc{P}'=\mc{P}\cap \Z^p\times \R^{n-p}$, the mixed-integer linear programming problem is then to compute the optimal value
\begin{equation}
 z^{IP}=\max_{x\in \mc{P'}}c^Tx,
\end{equation}

where $c\in \Q^{n}$ is a vector that defines the objective function.

%\section{Preliminary results and discussion}
We state some common strategies to find a solution of the general MILP
\begin{itemize}
 \item  Lower bounding method: A method for determining a lower bound on the objective function value of an optimal solution to a given subproblem.
 \item Upper bounding method: A method for determining an upper bound on the optimal solution value.
 \item Branching method: A procedure for partitioning a subproblem to obtain two or more children.
 \item Search strategy: A procedure for determining the search order.
\end{itemize}

\section{Theoretical framework and research plan}
In this project, we follow the ideas in \cite{cpt11} and \cite{cpt12} to further implement and research the cutting plane tree(CPT) algorithm. In \cite{cpt11}, 
a finitely convergent convex hull tree algorithm was developed, which is used to obtain the convex hull of general MILP and, moreover, construct a linear 
program that has the same optimal solution as the associate MILP. After combining this methods with standard notion of sequential cutting planes, 
they derive the CPT algorithm, which shows to converge to an integral optimal solution in finitely many iterations.

To implement the CPT algorithm, one should generate multi-term disjunctive cuts \todo{(seems hard..)}

%\section{Research plan}
After going into the details of the ideas, we will compare the effects of it with other preliminary algorithms. By test different problems under variety 
noncommercial or commercial software in \cite{soft05} and the CPT algorithm, we aim to know the efficiency of the algorithm in \cite{cpt11}. 

\bibliographystyle{plain} 
\bibliography{literature}
 
\end{document}
