\documentclass[a4paper,10pt]{article}

\input{conf.tex}
 
\title{Instruction on how to run the program}
\author{Chenxin Ma, Xi He}
 
\begin{document}
\maketitle
 
\section{Instruction} 

To run the program, we only need PuLP package installed properly. Then just go to "our code" file.

\begin{description}
\item{1.} If you want to run all the instances we have with both our implemented KBB algorithm and on PuLP, run the command "bash numerical.sh".
\item{2.} If you want to run on a particular set of instances on our implemented KBB algorithm, run the command "python main.py -f inst/knap$\_$n.inst.dat ". $n$ represents the number of variables of instances in this set. For example, "python main.py -f inst/knap$\_$20.inst.dat " would executes 20 dimensional instances. Additionally, the option -o would allow us to save the solution into files. For example, "python main.py -f inst/knap$\_$20.inst.dat -o sol/knap$\_$20.sol.dat" will execute the 20 dimensional instances and save the solution to a file under the folder "sol". 
\end{description}

 
\end{document}
