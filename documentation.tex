\documentclass[a4paper,10pt]{article}

\input{conf.tex}
 
\title{A Brief Instruction on How to Run the Program}
\author{Chenxin Ma, Xi He}
 
\begin{document}
\maketitle

The PuLP package of python is needed. After installing it properly, you can directly run the program both in one's computer or use the server \textit{polyps}. You can find all the related code in the \textit{code} directory. 

\begin{description}
\item{1.} To run all the instances we have with both our implemented KBB algorithm and PuLP, use the command \textit{bash numerical.sh}. And then the result will automatic save to a common file in \textit{oursol/res.txt}.
\item{2.} To run on a particular set of instances on our implemented KBB algorithm, use the command \textit{python main.py -f inst/knap\_n.inst.dat}, where $n$ represents the number of variables of instances in this set. For example, \textit{python main.py -f inst/knap\_20.inst.dat} would executes 20 dimensional instances. Additionally, the option $-o$ would allow us to save the solution into files. For example, \textit{python main.py -f inst/knap$\_$20.inst.dat -o sol/knap$\_$20.sol.dat} will execute the 20 dimensional instances and save the solution to a file under the folder \textit{sol}. 
\item{3.} To test the performance of pre-processing, you may use the similar commend as stated in item 1 and  2, but change the instances set to \textit{rand\_inst/rand\_n.dat}, where $n$ stands for the amount of the variables in the original problems.
\item{4.} For each step, you may also find corresponding code, where we use the function of the code as its name.
\end{description}

\end{document}
